\input{template/mmd-thesis-header}
\def\mytitle{Circuits}
\def\myauthor{Riccardo Orizio}
\def\languages{italian}
\def\latexmode{memoir}
\def\tocdepth{2}
\def\documenttitle{Progetto di Algoritmi e Strutture Dati}
\def\formattedtitle{Simulazione e Diagnostica di Circuiti Digitali}
\def\registrationnumber{91956}
\def\supervisor{Prof.ssa Marina Zanella}
\def\university{Università degli Studi di Brescia}
\def\faculty{Dipartimento di Ingegneria dell'Informazione}
\def\course{Corso di Laurea Magistrale in Ingegneria
Informatica}
\def\academicyear{201x-201x}
\input{template/mmd-natbib-plain}
\input{template/mmd-equation-mode}
\input{template/mmd-thesis-begin-doc}
\input{template/mmd-thesis-titlepage}
\input{template/mmd-thesis-index}

 % The empty line is important *in every file* 

\chapter{Introduzione}
\label{introduzione}

In questa documentazione verrà presentato il progetto relativo al corso di
Algoritmi e Strutture Dati, materia facente parte della laurea magistrale in
ingegneria informatica, tenuto dalla Professoressa Marina Zanella.

Verranno di seguito illustrati la tipologia di problema e, in modo più
approfondito, le metodologie risolutive utilizzate, trattando sia le strutture
dati per la gestione delle informazioni relative al problema sia gli algoritmi.
Al termine del documento verranno discussi i risultati ottenuti e tutti gli
accorgimenti presi durante lo sviluppo per ridurre i tempi di esecuzione e per
ottimizzare le prestazioni per risolvere il problema.

Il software è stato realizzato con il supporto della piattaforma web \emph{GitHub},
strumento molto utile per la gestione del progetto durante la sua creazione e
sviluppo; di conseguenza tutto il materiale può essere reperito nella repository
 \href{https://github.com/BoreaWolf/Circuits}{Circuits}.

TODO: Struttura zip

TODO: Compilazione


%   Progetto per ASD
%   Tutto presente su GitHub @
%   Struttura dello zip che verrà inviato
%   Come compilare


\chapter{Problema}
\label{problema}

Il problema è relativo a circuiti logici ed è suddiviso in due parti:
simulazione e diagnostica.

Nella parte di simulazione viene richiesto di simulare un circuito e di
studiarne le uscite, tenendo conto di eventuali problemi legati ad alcuni
componenti che potrebbero portare ad avere risultati reali differenti da quelli
teorici che ci si aspetterebbe.

Tutte le informazioni richieste per poter creare e studiare il circuito ed il
suo comportamento vengono fornite tramite dei file esterni dati in input al
programma, descritti qui di seguito:

\begin{enumerate}
\item Circuito: descrizione dell'intero circuito con tutti i suoi componenti,
 suddivisi in terminali di ingresso ed uscita e porte logiche interne.

\item Ingressi: valori binari rappresentativi gli ingressi del circuito.

\item Guasti: insieme dei componenti del circuito con guasti, indicativi anche del
 tipo di guasto ai quali sono soggetti.

\end{enumerate}

Il programma, una volta terminato, genererà diversi file in uscita, i quali
verranno utilizzati per la parte diagnostica, descriventi diverse componenti del
circuito appena studiato:

\begin{enumerate}
\item Uscite teoriche: valori di uscita di tutti i componenti del circuito nel
 caso in cui non ci sia alcun tipo di guasto nei suoi componenti.

\item Uscite reali: valori di uscita di tutti i componenti del circuito tenendo
 conto dei guasti dei componenti letti in ingresso.

\item Risultato comparativo delle uscite: indicheremo con OK quei componenti che
 hanno uscita reale equivalente all'uscita teorica, mentre indicheremo con KO
 quelle uscite che differiscono tra loro.

\item Coni dei terminali d'uscita: lista dei componenti, esclusi i terminali
 d'input, dai quali dipende il valore d'uscita.

\end{enumerate}

Nella parte diagnostica viene richiesto di{\ldots}

\chapter{Simulazione}
\label{simulazione}

La prima parte del progetto riguarda la simulazione di un circuito digitale bla
bla bla.

\chapter{Diagnostica}
\label{diagnostica}

La seconda parte del progetto riguarda diagnosticare particolari richieste
fatte dall'utente, diagnostiche che verranno svolte su circuiti noti e già
processati nella parte di simulazione.

\input{template/mmd-thesis-footer}

\end{document}
